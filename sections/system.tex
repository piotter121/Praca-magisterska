\chapter{Projekt systemu do prowadzenia projektów}

Rozdział ten opisuje autorski system do prowadzenia projektów w trzech wersjach, każda z nich różni się zastosowanym modelem i bazą danych.
Powstały dwie wersje odpowiednio dla baz Apache Cassandra i MongoDB, oraz trzeci wariant z modelem hybrydowym, gdzie część danych jest przechowywana w bazie relacyjnej, a część w bazie nierelacyjnej.
Główna część systemu powstała jako rezultat pracy inżynierskiej.
Niniejsza praca bazuje na nim rozszerzając nieznacznie jego funkcjonalność w celu dogłębniejszej analizy możliwości porównywanych baz.

\section{Specyfikacja wymagań funkcjonalnych}

Jedną z najważniejszych faz rozwoju oprogramowania jest wyspecyfikowanie wymagań.
Jest to część inżynierii oprogramowania dostarczająca nam środków i metod umożliwiających zebranie na temat funkcjonalności tworzonej przez nas aplikacji lub systemu komputerowego.
Popełnienie błędu w tej fazie projektu jest najbardziej kosztowne, ponieważ jest to początkowa faza i błędy w niej popełnione propagują się na kolejne fazy.

\subsection{Role i uprawniania użytkowników w systemie}

W opisywanym systemie do prowadzenie projektów można wyróżnić następujące role i powiązane z nimi uprawnienia:
\begin{itemize}
    \item \textbf{Użytkownikiem} jest każdy, kto posiada zarejestrowane konto w aplikacji.
    \item \textbf{Administratorem projektu} staje się użytkownik aplikacji w momencie, gdy stworzy nowy projekt.
    Daje mu to uprawnienia do:
    \begin{itemize}
        \item dodawania nowych zadań w projekcie,
        \item usuwania istniejących zadań,
        \item usunięcia administrowanego projektu.
    \end{itemize}
    \item \textbf{Uczestnikiem zadania} staje się użytkownik, który został przypisany do zadania przez administratora.
    Dzięki temu zyskuje uprawnienia do:
    \begin{itemize}
        \item dodawania nowego pliku do zadania,
        \item pobrania wybranej wersji pliku,
        \item zaznaczenia pliku do zatwierdzenia przez administratora zadania,
        \item zapisania nowej wersji pliku,
        \item wyświetlenia szczegółowych informacji o wersji pliku.
    \end{itemize}
    \item \textbf{Administratorem zadania} zostaje użytkownik aplikacji, który został wybrany do tego przez administrator projektu podczas tworzenia zadania.
    Otrzymuje on wtedy następujące przywileje:
    \begin{itemize}
        \item możliwość przypisania/usunięcia uczestnika do/z administrowanego zadania,
        \item możliwość zatwierdzenia ostatecznej wersji pliku,
        \item możliwość usunięcia pliku.
    \end{itemize}
    Oprócz tych uprawnień użytkownik pełniący tą rolę otrzymuje wszystkie uprawniania uczestnika projektu.
\end{itemize}

Jeden użytkownik może pełnić wiele z wyżej wymienionych ról.
Pełniąc je dziedziczy wymienione uprawnienia. 

\subsection{Przypadki użycia}

Przypadki użycia opisywanego systemu zostały podzielone na dwie grupy funkcjonalne.
Ma to na celu łatwiejsze zrozumienie oferowanej przez system funkcjonalności.

\subsubsection{Funkcjonalność związana z projektami i zadaniami}

Zbiór przypadków użycia związany z zarządzaniem projektami i zadaniami w systemie:
\begin{enumerate}
    \item \textbf{Utworzenie projektu} -- każdy użytkownik systemu może utworzyć i zarządzać swoimi projektami.
    \item \textbf{Wyświetlenie listy projektów}, w których uczestniczy użytkownik jest wyświetlane tuż po zalogowaniu się do aplikacji.
    \item \textbf{Wyświetlenie szczegółowych informacji o projekcie wraz z listą zadań} -- użytkownik może wyświetlić informacje o projekcie jeżeli jest uczestnikiem projektu tzn. jest administratorem projektu, lub administratorem lub uczestnikiem zadania.
    \item \textbf{Wyświetlenie szczegółowych informacji o zadaniu wraz z listą plików} -- użytkownik może wyświetlić informacje o zadaniu jeżeli jest przypisany do niego, lub jest jego administratorem.
    \item \textbf{Dodanie nowego zadania do projektu} -- uprawniony do tego jest administrator projektu.
    \item \textbf{Usunięcie zadania} -- uprawniony do tego jest administrator projektu.
    \item \textbf{Usunięcie projektu} -- uprawniony do tego jest administrator projektu.
    \item \textbf{Przypisanie nowego uczestnika do zadania} -- ta funkcja jest dostępna dla administratora zadania.
    \item \textbf{Usunięcie uczestnika z zadania} -- ta funkcja jest dostępna dla administratora zadania.
\end{enumerate}

\subsubsection{Funkcjonalność związana z plikami i ich wersjami}

Przypadki użycia związane z wszystkimi czynnościami, które można wykonać w związku z plikami w~obrębie systemu:
\begin{enumerate}
    \item \textbf{Wyszukiwanie pełnotekstowe wśród plików zadania/projektu} jest rozszerzeniem funkcjonalności systemu w porównaniu do systemu powstałego w pracy inżynierskiej.
    Ten przypadek użycia ma na celu zbadanie możliwości i wsparcia wyszukiwania pełnotekstowego przez bazy danych.
    Wyszukiwanie to jest dostępne dla uczestników projektów i zadań.
    Można je uruchomić z poziomu widoku listy zadań -- wtedy zostaną przeszukane wszystkie pliki zadań, do których użytkownik ma dostęp.
    Wyszukiwanie z widoku listy plików w zadaniu ogranicza przeszukiwane pliki do tych, które znajdują się w zadaniu.
    Po wyświetleniu listy plików dopasowanych do wyszukiwanej frazy można wybrać jeden z nich i przejść do szegółowych informacji o nim.
    \item \textbf{Zatwierdzenie pliku} -- przypadek użycia przeznaczony dla administratora zadania. 
    Może być zrealizowany gdy system wyświetla listę plików przypisanych do zadania. 
    Po wybraniu pozycji z tej listy administrator wybiera opcję zatwierdzenia. 
    Po tej czynności plik posiada status zatwierdzony i nie może można go już edytować.
    \item  \textbf{Usunąć plik} może tylko administrator zadania. 
    Wraz z plikiem usuwane są wszystkie zapisane wersje. 
    W celu usunięcia pliku administrator wybiera pozycję z listy plików i~wybiera opcję usunięcia.
    \item \textbf{Wyświetlenie szczegółów pliku z listą wersji} -- uprawniony do tego jest uczestnik zadania.
    Może tego dokonać w poziomu widoku listy plików.
    \item \textbf{Dodanie nowego pliku} -- funkcjonalność dostępna dla każdego uczestnika zadania. 
    \item \textbf{Pobranie wersji pliku} -- dostępne dla uczestnika zadania. 
    Po przejściu do widoku listy wersji pliku może wybrać opcję pobrania konkretnej wersji.
    \item \textbf{Zaznaczenie pliku do zatwierdzenia przez administratora zadania} -- prosty przypadek użycia dostępny dla uczestnika zadania.
    Po wyświetleniu  szczegółowego widoku informacji o pliku może wybrać opcję oznaczenia pliku do zatwierdzania.
    \item \textbf{Zapisanie nowej wersji pliku} jest dostępne dla uczestnika zadania z poziomu widoku szczegółowych informacji o pliku.
    \item \textbf{Wyświetlenie szczegółowych informacji o wersji pliku} -- opcja dostępna dla uczestników zadania.
\end{enumerate}

\section{Architektura systemu}

System został zaprojektowany i zaimplementowany zgodnie z architekturą wielowarstwową jako aplikacja internetowa.
Składa się z dwóch aplikacji: klienta przeglądarkowego i serwera aplikacyjnego.
Można wyróżnić w nim następujące warstwy:
\begin{itemize}
    \item \textbf{warstwa dostępu do danych} -- nadaje poziom abstrakcji ukrywając technologię stojącą za przechowywaniem danych systemu.
    Na jej poziomie następują bezpośrednie połączenia z konkretną bazą danych i komunikacją z nią.
    Dzięki niej wymiana bazy danych sprowadza się do zmian we właśnie tej warstwie.
    \item \textbf{warstwa logiki biznesowej} -- odpowiedzialna za główną logikę aplikacji, koordynuje jej pracę. 
    Ponadto jest odpowiedzialna za przetwarzanie i przekazywanie danych między warstwą dostępu do danych, a warstwą kontrolerów.
    \item \textbf{warstwa kontrolerów} -- najbardziej zewnętrzna warstwa w serwerze aplikacyjnym, wystawia sieciowy interfejs programowania aplikacji (ang.~\textit{WebAPI}) dla klienta przeglądarkowego.
    Jest odpowiedzialna za przyjmowanie i odpowiadanie żądania od niego poprzez wywoływanie odpowiednich procedur warstwy logiki biznesowej.
    \item \textbf{warstwa prezentacji} -- klient przeglądarkowy, stanowi główny interfejs dla użytkownika systemu.
\end{itemize}

\section{Model danych}

\subsection{Apache Cassandra + Elasticsearch}

\subsection{MongoDB}

\subsection{Model hybrydowy}