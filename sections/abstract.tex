\renewcommand{\abstractname}{STRESZCZENIE}
\begin{abstract}

Celem niniejszej pracy dyplomowej jest analiza porównawcza i wybór optymalnej technologii przechowywania danych spośród rozwiązań nierelacyjnych w systemie do prowadzenia projektów.
W takim systemie ważne jest zachowanie wysokiej dostępności przy spójności danych.
Analiza została dokonana w kilku etapach.

Pierwszym z nich była przedstawienie i analiza sposobu działania wybranych wydajnych baz nierelacyjnych.
Porównane zostały sposoby przechowywania i organizacji danych, i kluczowe mechanizmy działania tych systemów.
Opisany został również sposób ich działania w środowisku rozproszonym.
Ważnym kryterium wyboru są biblioteki klienckie i narzędzia dostępowe, które również zostały opisane.
Przydatną i niedostępną zbyt powszechnie w relacyjnych systemach jest możliwość indeksowania i wyszukiwania pełnotekstowego.

Kolejnym etapem była implementacja i przedstawienie budowy systemu do prowadzenia projektów, który powstał w trzech wersjach różniących się zastosowanymi modelami danych.
Opisana została specyfikacja wymagań funkcjonalnych z wydzielonymi rolami w systemie i odpowiednio pogrupowanymi przypadkami użycia.
Zaprezentowano również architekturę systemu z uwzględnieniem poszczególnych jego warstw oraz wykorzystany stos technologiczny.

Na koniec dokonano analizy wydajnościowej za pomocą syntetycznych testów z wykorzystaniem programu JMeter.
Po wyznaczeniu porównywanych metryk i scenariuszów testowych przystąpiono do testów.
Po testach wyniki zostały przeanalizowane pod względem czasu odpowiedzi aplikacji i szybkości przetwarzania żądań.

Praca ta stanowi wstęp do dalszego rozwoju powstałego systemu.
Po dokonaniu optymalnego wyboru bazy danych możliwy jest jego podział na mikroserwisy.
Pozwoliłoby to na sprawdzenie zachowania aplikacji w środowisku rozproszonym.
\\[5mm]

\textbf{Słowa kluczowe:} aplikacja, analiza, NoSQL, projekt, testy 

\end{abstract}
\clearpage

\renewcommand{\abstractname}{ABSTRACT}
\begin{abstract}
\begin{center}
\textbf{Comparative analysis of efficient non-relational databases in a project management system}
\end{center} 

The aim of this diploma thesis is comparative analysis and selection of optimal data storage technology among non-relational solutions in the project management system.
In such a system, it is important to maintain high availability with data consistency.
The analysis was made in several stages.

The first step was presentation and analysis of the way chosen efficient non-relational bases work.
The methods of storing and organising data, and key mechanisms of these systems have been compared.
The way they work in a distributed environment is also outlined.
An important selection criterion are client libraries and access tools, which are also described.
Useful and not available too commonly in relational systems is the ability to index and search in full-text.

The next step was the implementation and presentation of the construction of the project management system, which was created in three versions differing with the data models used.
A specification of functional requirements with separated roles in the system and grouped use cases has been described.
The system architecture was also presented, taking into account its individual layers and the technological stack used.

Finally, a performance analysis was performed using synthetic tests and JMeter application.
After the designation of comparable metrics and test scenarios, tests were started.
After the tests, the results were analysed in terms of the response time of the application and the throughput of request processing.

This diploma thesis is an introduction into further development of created system.
After selection of optimal database, it is possible to divide it into microservices.
This would allow to check the behaviour of the application in a distributed environment.
\\[5mm]

\textbf{Keywords:} application, analysis, NoSQL, project, tests

\end{abstract}

\clearpage