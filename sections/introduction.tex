\chapter{Wprowadzenie}

Prowadzenie projektów wymaga równoległego prowadzenia ich dokumentacji. 
Wraz ze wzrostem jej obszerności okazuje się jak ważną rolę pełni dobry system do prowadzenia takiej dokumentacji.
Dzięki niemu można zaoszczędzić wiele czasu i zwiększyć efektywność pracy zespołu.
Nie tylko pełni on role głównego magazynu do przechowywania i organizacji cyfrowego archiwum, ale również daje możliwość podglądu zmian dokonywanych przez członków grupy.

Efektem mojej pracy inżynierskiej był projekt i implementacja prototypu prostego systemu do obiegu dokumentów w projektach. 
Aplikacja ta została wykonana jako trójwarstwowa aplikacja internetowa.
System ten rozwiązywał następujące problemy:
\begin{itemize}
    \item zarządzenie strukturą projektu: podział na podzadania i przydzielanie do nich odpowiednich uczestników,
    \item śledzenie, kto dokonał, jakich zmian,
    \item śledzenie, która wersja dokumentu jest najnowsza.
\end{itemize}

System ten, z powodu ograniczeń czasowych, nie posiadał rozbudowanej funkcjonalności, stanowił punkt startowy dla narzędzia wspomagającego obieg dokumentów w projektach.
Warstwa przechowywania danych została zaprojektowana i zaimplementowana z wykorzystaniem relacyjnej bazy danych, o czym zdecydowała dostępność i popularność obecnych na rynku rozwiązań. 

Bazy danych z tej rodziny dominują na rynku od wielu lat \cite{DBEnginesRangking}, co jest anomalią w świecie ciągle zmieniającej się branży informatycznej.
Mimo wielu zalet posiadają one swoje ograniczenia m. in. w postaci sztywnego schematu danych.
Wraz ze wzrostem złożoności modelu relacyjnej bazy danych rośnie również czas operacji i zapytań do takiej bazy.
Rosnące wymagania związane z~rozwojem aplikacji internetowych spowodowały powstanie wielu nowych typów baz danych, zwanych \textit{NoSQL}.
Nowe rozwiązania w różny sposób i w różnym kierunku odchodzą od klasycznego modelu relacyjnego.
Bazy te rezygnują z postaci relacyjnej i związanego z nimi iloczynu kartezjańskiego na rzecz innych modeli danych.
Czasami struktury danych używane przez nie są również postrzegane jako bardziej elastyczne.
Istnieją różne podejścia do klasyfikacji baz danych NoSQL.
Podstawowa klasyfikacja ze względu na ich model danych obejmuje bazy \cite{NoSQLKompendiumWiedzy}:
\begin{itemize}
    \item \textbf{klucz-wartość}, wykorzystujące tablice asocjacyjne do przechowywania danych,
    \item \textbf{dokumentowe}, przechowujące dane w dokumentach, zapisanych w różnych standardach formatowania (XML, JSON, YAML), które są pogrupowane w kolekcje,
    \item \textbf{kolumnowe}, przechowujące dane w grupach kolumn, które można interpretować jako dwuwymiarowe bazy typu klucz-wartość,
    \item \textbf{grafowe} przeznaczone dla danych, których związki są dobrze reprezentowane jako graf.
\end{itemize}

Dynamiczny rozwój nierelacyjnych baz danych pokazuje, że wymagania stawiane przed nowymi projektami są na tyle duże, że nie można dłużej mówić o uniwersalnym rozwiązaniu każdego problemu, jakim do niedawna mogły wydawać się bazy relacyjne.

\section*{Cel i zakres pracy}

Głównym celem niniejszej pracy dyplomowej jest analiza porównawcza wybranych baz nierelacyjnych w kontekście aplikacji do prowadzenia projektów i dokumentacji z nimi związanych. 
Analiza posłużyła do wyboru optymalnego rozwiązania, które pozwoliłoby na wydajną pracę omawianej aplikacji i nie hamowało by rozszerzania jej funkcjonalności.

Do celów porównawczych powstał system obiegu dokumentów w projektach w trzech wersjach, w~zależności od wykorzystanej bazy danych:
\begin{itemize}
    \item korzystający z bazy dokumentowej MongoDB,
    \item korzystający z bazy kolumnowej Apache Cassandra,
    \item korzystający z bazy kolumnowej Apache Cassandra i bazy relacyjnej PostgreSQL.
\end{itemize}
Funkcjonalność systemu została dodatkowo rozszerzona o wyszukiwanie pełnotekstowe w przechowywanych w systemie plikach.

Zakres pracy obejmują również testy wydajnościowe powstałych wersji systemu, których wyniki zostały poddane dogłębnej analizie pod kątem wpływu modelu i technologii bazy danych na szybkość wykonywania podstawowych operacji w systemie.