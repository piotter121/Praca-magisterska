\chapter{Podsumowanie}

W tej pracy dyplomowej zostały porównane dwie wydajne nierelacyjne bazy danych -- Apache Cassandra i MongoDB -- w kontekście ich użycia w systemie służącym do prowadzenia projektów z prostym wersjonowaniem dokumentów.
Na początku zostały przeanalizowane sposoby organizacji danych przez poszczególne bazy.
Tabelaryczny sposób organizacji danych w Cassandrze może przypominać relacyjne bazy danych, lecz jest pozbawiony wszystkiego, co może spowolnić operacje w tego rodzaju bazach -- kosztownych operacji złączeń tabel, więzów integralności i~wyzwalaczy.
Z kolei dokumenty w MongoDB doskonale odzwierciedlają model obiektowy i mogą z łatwością reprezentować dowolny obiekt domeny biznesowej.
Przydatną opcją jest możliwość indeksowania dowolnych pól dokumentów w kolekcji, nawet tych zagnieżdżonych, dzięki czemu można przyspieszyć często używane zapytania.

Opisowi zostały poddane również procesy zapisu i odczytu danych oraz sposób pracy w~środowisku rozproszonym.
Pomimo, że każda z tych baz została stworzona do pracy w tym środowisku reprezentują one dwa trochę różniące się podejścia do tego tematu.
MongoDB z jednym węzłem nadrzędnym, przyjmującym operacje zapisów, w większym stopniu stawia na spójność danych, ponieważ węzeł ten zarządza replikami danych.
W inny sposób działa Cassandra, która w~środowisku \textit{peer-to-peer}, gdzie każdy węzeł może przyjąć operację zapisu, dąży do jak największej dostępności i bezawaryjności.

Obie bazy danych posiadają bogaty zbiór narzędzi dostępowych i bibliotek klienckich, dzięki czemu wykorzystanie ich w rozwiązaniach różnych problemów jest o wiele łatwiejsze.
\textit{Cassandra Query Language} jest znacznym ułatwieniem dla osób zaczynających poznawanie nierelacyjnych baz danych, a znających już \textit{SQL}.
Bez projektu Spring Data wdrożenie ich do omawianego w~pracy systemu było dużo trudniejsze i wolniejsze.

W tym samym rozdziale zostało jeszcze opisane wsparcie dla wyszukiwania pełnotekstowego. 
MongoDB posiada natywną możliwość utworzenia indeksu tekstowego na danej kolekcji dokumentów.
Z kolei Apache Cassandra wymaga integracji z wtyczką silnika wyszukiwania pełnotekstowego Elasticsearch.
Jednakże wraz z tą wtyczką zyskuje większy wachlarz możliwości analizy i wyszukiwania indeksowanych w systemie dokumentów.

W trzecim rozdziale został opisany projekt zaimplementowanego w trzech wersjach systemu.
Pierwsze dwie wersje opierają się na wspomnianych wcześniej bazach NoSQL.
Trzecia zaś próbuje połączyć w jednym modelu danych relacyjną bazę PostgreSQL i nierelacyjną Cassandrę.
W tym samym rozdziale znajduje się dokładniejszy opis każdego z tych modeli danych.

Czwarty rozdział zawiera analizę wydajnościową powstałych wersji systemu.
W celu przeprowadzenia jej zostały wykonane syntetyczne testy obciążeniowe za pomocą programu JMeter.
Ograniczenia platformy testowej nie pozwalała na odtworzenie środowiska rozproszonego tych.
Scenariusze testowe bazowały na przypadkach użycia systemu.
Pierwszy scenariusz obejmował operacje intensywnie korzystające z zapisu danych do systemu.
Zwycięsko z niego wyszedł system z modelem hybrydowym na co przełożyła się m. in. struktura próbek testowych, którą stanowiły w dużej mierze operacje dodania nowego pliku do system.
O szybkim wykonaniu ich zdecydowały brak redundancji danych i potrzeby ich zaktualizowania w innych miejscach, oraz zapis przesyłanego pliku do Cassandry, co jet czynnością szybszą niż zapis do PostgreSQL.

Drugi scenariusz składał się z dużej ilości operacji odczytu danych z praktycznie całego modelu danych.
Tym razem bezkompromisowo wygrała wersja systemu podłączona do bazy Apache Cassandra.
Wąskie gardło modelu hybrydowego stanowiła część relacyjna.

\section*{Wnioski i potencjalne kierunki rozwoju}

Szybki postęp technologiczny i bardzo duża ilość napływających zewsząd danych każą nam zrewidować poglądy na wciąż bardzo popularne relacyjne bazy danych.
Skłaniając się ku wysoko skalowalnym rozwiązaniom NoSQL trzeba mieć na uwadze, że będziemy musieli zrezygnować (zgodnie z twierdzeniem Brewera o CAP~\cite{BrewerTheorem}) albo ze spójności danych, albo z bardzo wysokiej dostępności systemu.
Nie ma wśród nich rozwiązań uniwersalnych.
Wybór też jest w dużej mierze zależny od modelu domeny systemu. 
W skomplikowanym modelu, przechowywanym w bazie kolumnowej takiej jak Cassandra, zbyt duży stopień denormalizacji i niespójności danych może w~pewnym momencie stać się nie do zaakceptowania.
Do takiej sytuacji może dojść w omówionym w pracy systemie w raz z jego rozwojem.
Model bazujący na MongoDB daje szersze pole możliwości rozwoju.
Dodatkowo przeprowadzona analiza wydajnościowa pokazuje, że jest rozwiązanie wystarczająco wydajne przy zachowaniu bardzo ważnej spójności danych.
Wśród alternatyw jest również możliwość rozbicia systemu na mikroserwisy, gdzie każdy z nich posiadałbym własną, odpowiednią dla jego potrzeb bazę danych.
